\documentclass{article}

\usepackage[utf8]{inputenc}
\usepackage{amsmath,amsthm,amssymb}

\title{Fibonacci Closed-Form Solution}
\author{Robert Mitchell \\
        robert.mitchell36@gmail.com}
\date{August 2017}

\begin{document}

\maketitle

\noindent
\textbf{Proposition:} $ \forall n \in \mathbb{Z}^{\geq 0} $, where $ \phi = \frac{1 + \sqrt{5}}{2} $, and $ \psi = \frac{1 - \sqrt{5}}{2} $
\begin{enumerate}
    \item $ \text{Fib}(n) $ is the closest integer to $ \frac{\phi^n}{\sqrt{5}} $
    \item $ \text{Fib}(n) = \frac{\phi^n - \psi^n}{\sqrt{5}} $ \\
\end{enumerate}

\noindent
\textbf{Proof of 1:} By induction on $ n $:

\textbf{Base cases: $ n \in \{0, 1\} $}

Take $ n = 0 $: $ \text{Fib}(0) = 0 $, and $ \frac{\phi^0}{\sqrt{5}} = \frac{1}{\sqrt{5}} $, and $ [ \frac{1}{\sqrt{5}} ] = 0 $ as desired.

Take $ n = 1 $: $ \text{Fib}(1) = 1 $, and $ \frac{\phi}{\sqrt{5}} = \frac{1 + \sqrt{5}}{2\sqrt{5}} = \frac{1}{2\sqrt{5}} + \frac{1}{2} = \frac{1}{2}(\frac{1}{\sqrt{5}} + 1) $, \\
and since $ [\frac{1}{\sqrt{5}} + 1] = 1 $, it follows that $ [\frac{\phi}{\sqrt{5}}] = 1 $ as desired. \\

\textbf{Inductive step:} Take arbitrary $ n \in \mathbb{N} $, and assume $ [\frac{\phi^n}{\sqrt{5}}] = \text{Fib}(n) $ and $ [\frac{\phi^{n-1}}{\sqrt{5}}] = \text{Fib}(n-1) $.

It suffices to show that $$ \text{Fib}(n + 1) = [\frac{\phi^{n+1}}{\sqrt{5}}]. $$

Given the definition of $ \text{Fib}(n + 1) $ as $ \text{Fib}(n) + \text{Fib}(n - 1) $, it follows immediately from the inductive hypothesis that $ \text{Fib}(n + 1) = [\frac{\phi^n}{\sqrt{5}}] + [\frac{\phi^{n-1}}{\sqrt{5}}] $.
Therefore, it suffices to show that $$ [\frac{\phi^{n+1}}{\sqrt{5}}] = [\frac{\phi^n}{\sqrt{5}}] + [\frac{\phi^{n-1}}{\sqrt{5}}] $$
or, equivalently,
$$ |\frac{\phi^{n+1} - \phi^n - \phi^{n-1}}{\sqrt{5}}| < \frac{1}{2} $$

Factoring out common terms,

$$ \frac{\phi^{n-1}}{\sqrt{5}} | \phi^2 - \phi - 1 | < \frac{1}{2} \rightarrow |\phi^2 - \phi - 1| < \frac{\sqrt{5}}{2\phi^{n-1}} $$

By definition, $ \phi = \frac{1+\sqrt{5}}{2} $, so the above is equivalently
$$ |\frac{1 + 2\sqrt{5} + 5}{4} - \frac{1 + \sqrt{5}}{2} - 1| < \frac{\sqrt{5}}{2\phi^{n-1}} $$
Thus, $$ |\frac{6 + 2\sqrt{5} - 2 - 2\sqrt{5} - 4}{4}| < \frac{\sqrt{5}}{2\phi^{n-1}} $$
which is, by arithmetic, $ 0 < \frac{\sqrt{5}}{2\phi^{n-1}} $, which is obviously true, since $ \forall n \in \mathbb{N}, \phi^n > 0 $.

Therefore, given $ [\frac{\phi^n}{\sqrt{5}}] = \text{Fib}(n) $ and $ [\frac{\phi^{n-1}}{\sqrt{5}}] = \text{Fib}(n-1) $, it is true that $ \text{Fib}(n + 1) = [\frac{\phi^{n+1}}{\sqrt{5}}] $, and therefore, by induction, $\forall n \in \mathbb{N}, \text{Fib}(n) = [\frac{\phi^n}{\sqrt{5}}]$

\hfill \textbf{Q.E.D.} \\

Before proving the second portion of the proposition, let us prove this Lemma which will make the proof of part 2 quite simple. \\


\noindent
\textbf{Lemma:} $ \forall n \in \mathbb{N} $, with $\phi$ and $\psi$ defined as in the above proposition,
\begin{enumerate}
    \item $ \phi^n = \phi^{n-1} + \phi^{n-2} $
    \item $ \psi^n = \psi^{n-1} + \psi^{n-2} $
\end{enumerate}


\noindent
\textbf{Proof of 1:} By induction on $ n $:

\textbf{Base case: $n = 2$} By direct computation,
$$ \phi^2 = \frac{(1+\sqrt{5})^2}{4} = \frac{1}{4}(1+2\sqrt{5}+5) = \frac{3 + \sqrt{5}}{2} $$
And $$ \phi + \phi^0 = \frac{1+\sqrt{5}}{2} + 1 = \frac{3 + \sqrt{5}}{2} = \phi^2 $$
Therefore, for $ n = 2 $, $ \phi^n = \phi^{n - 1} + \phi^{n - 2} $. \\

\textbf{Inductive step:} Take arbitrary $ n \in \mathbb{N} $, where $ n > 2 $, and assume $ \phi^n = \phi^{n - 1} + \phi^{n - 2} $. We desire to show that $$ \phi^{n + 1} = \phi^n + \phi^{n - 1} $$

Observe that $$ \phi^{n + 1} = \phi\phi^n = \phi(\phi^{n - 1} + \phi^{n - 2}) $$ by assumption, and that by distribution this is equal to $$ \phi^n + \phi^{n - 1} $$ as desired.
Therefore, $ \forall n \in \mathbb{N} $, where $ n \geq 2 $, $ \phi^n = \phi^{n - 1} + \phi^{n - 2} $.

\hfill \textbf{Q.E.D.} \\

\noindent
\textbf{Proof of 2:} By induction on $ n $:

\textbf{Base case: $n = 2$} By direct computation,
$$ \psi^2 = \frac{(1-\sqrt{5})^2}{4} = \frac{1}{4}(1-2\sqrt{5}+5) = \frac{3 - \sqrt{5}}{2} = 1 + \frac{1 - \sqrt{5}}{2} = \psi^0 + \psi $$
Therefore, for $ n = 2 $, $ \psi^n = \psi^{n - 1} + \psi^{n - 2} $. \\

\textbf{Inductive step:} Take arbitrary $ n \in \mathbb{N} $, where $ n > 2 $, and assume $ \psi^n = \psi^{n - 1} + \psi^{n - 2} $. We desire to show that $$ \psi^{n + 1} = \psi^n + \psi^{n - 1} $$

Observe that $$ \psi^{n + 1} = \psi\psi^n = \psi(\psi^{n - 1} + \psi^{n - 2}) $$ by assumption, and that by distribution this is equal to $$ \psi^n + \psi^{n - 1} $$ as desired.
Therefore, $ \forall n \in \mathbb{N} $, where $ n \geq 2 $, $ \psi^n = \psi^{n - 1} + \psi^{n - 2} $.


\hfill \textbf{Q.E.D.} \\


Now, returning to the second part of the Proposition, \\

\noindent
\textbf{Proof of 2:} By induction on $ n $:

\textbf{Base case: $n = 0$} By direct computation,
$ \text{Fib}(0) = 0 $, and $$ \frac{\phi^0 - \psi^0}{\sqrt{5}} = \frac{1 - 1}{\sqrt{5}} = 0 $$.
Thus, for $ n = 0 $, $ \text{Fib}(n) = \frac{\phi^n - \psi^n}{\sqrt{5}} $

\textbf{Base case: $n = 1$} By direct computation,
$$ \text{Fib}(1) = 1 = \frac{1 + \sqrt{5} - 1 + \sqrt{5}}{2\sqrt{5}} = \frac{\phi - \psi}{\sqrt{5}} $$

\textbf{Base case: $n = 2$} By direct computation,
$ \text{Fib}(2) = \text{Fib}(1) + \text{Fib}(0) = 1 $, and $$ \frac{\phi^2 - \psi^2}{\sqrt{5}} = \frac{1}{\sqrt{5}}(\phi + \phi^0 - \psi - \psi^0) = \frac{1}{\sqrt{5}}(\frac{1 + \sqrt{5} - (1 - \sqrt{5})}{2}) = \frac{2\sqrt{5}}{2\sqrt{5}} = 1 $$
Therefore, $ \text{Fib}(2) = \frac{\phi^2 - \psi^2}{\sqrt{5}} $, as desired.


\textbf{Inductive step:} Take arbitrary $ n \in \mathbb{N} $, where $ n > 2 $, and assume $ \text{Fib}(n) = \frac{\phi^n-\psi^n}{\sqrt{5}} $ and $ \text{Fib}(n - 1) = \frac{\phi^{n - 1} - \psi^{n - 1}}{\sqrt{5}} $. \\

Observe that, since $ \text{Fib}(n + 1) = \text{Fib}(n) + \text{Fib}(n - 1) $, $$ \text{Fib}(n + 1) = \frac{\phi^n - \psi^n}{\sqrt{5}} + \frac{\phi^{n - 1} + \psi^{n - 1}}{\sqrt{5}} = \frac{1}{\sqrt{5}}(\phi^n + \phi^{n - 1} - \psi^n - \psi^{n - 1}) $$ Which, by the above Lemma, is equal to $$ \frac{\phi^{n + 1} - \psi^{n - 1}}{\sqrt{5}} $$ as desired.

Therefore, $$ \forall n \in \mathbb{N}, \text{Fib}(n) = \frac{\phi^n - \psi^n}{\sqrt{5}} $$

\hfill \textbf{Q.E.D.} \\

\end{document}
